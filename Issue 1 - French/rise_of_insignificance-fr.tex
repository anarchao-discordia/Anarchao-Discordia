\chapter{Stopper la montée de l'insignifiance}
\chapterauthor{Cornelius Castoriadis}

\emph{Ce texte est issue d'une interview par Daniel Mermet pour le programme radiophonique ``L\`a-bas si j'y suis" datant de 1996. Il a ensuite était publié dans \emph{Le Monde Diplomatique}, Août 1998.\\}

Ce qui caractérise le monde contemporain ce sont, bien sûr, les crises, les contradictions, les oppositions, les fractures, mais ce qui me frappe surtout, c’est l’insignifiance. Prenons la querelle entre la droite et la gauche. Elle a perdu son sens. Les uns et les autres disent la même chose. Depuis 1983, les socialistes français ont fait une politique, puis M. Balladur a fait la même politique ; les socialistes sont revenus, ils ont fait, avec Pierre Bérégovoy, la même politique ; M. Balladur est revenu, il a fait la même politique ; M. Chirac a gagné l’élection de 1995 en disant : ``Je vais faire autre chose" et il a fait la même politique.\\
Les responsables politiques sont impuissants. La seule chose qu’ils peuvent faire, c’est suivre le courant, c’est-à-dire appliquer la politique ultralibérale à la mode. Les socialistes n’ont pas fait autre chose, une fois revenus au pouvoir. Ce ne sont pas des politiques, mais des politiciens au sens de micropoliticiens. Des gens qui chassent les suffrages par n’importe quel moyen. Ils n’ont aucun programme. Leur but est de rester au pouvoir ou de revenir au pouvoir, et pour cela ils sont capables de tout.\\
Il y a un lien intrinsèque entre cette espèce de nullité de la politique, ce devenir nul de la politique et cette insignifiance dans les autres domaines, dans les arts, dans la philosophie ou dans la littérature. C’est cela l’esprit du temps. Tout conspire à étendre l’insignifiance.\\
La politique est un métier bizarre. Parce qu’elle présuppose deux capacités qui n’ont aucun rapport intrinsèque. La première, c’est d’accéder au pouvoir. Si on n’accède pas au pouvoir, on peut avoir les meilleures idées du monde, cela ne sert à rien ; ce qui implique donc un art de l’accession au pouvoir. La seconde capacité, c’est, une fois qu’on est au pouvoir, de savoir gouverner.\\
Rien ne garantit que quelqu’un qui sache gouverner sache pour autant accéder au pouvoir. Dans la monarchie absolue, pour accéder au pouvoir il fallait flatter le roi, être dans les bonnes grâces de Mme de Pompadour. Aujourd’hui dans notre ``pseudo-démocratie", accéder au pouvoir signifie être télégénique, flairer l’opinion publique.\\
Je dis ``pseudo-démocratie" parce que j’ai toujours pensé que la démocratie dite représentative n’est pas une vraie démocratie. Jean-Jacques Rousseau le disait déjà : les Anglais croient qu’ils sont libres parce qu’ils élisent des représentants tous les cinq ans, mais ils sont libres un jour pendant cinq ans, le jour de l’élection, c’est tout. Non pas que l’élection soit pipée, non pas qu’on triche dans les urnes. Elle est pipée parce que les options sont définies d’avance. Personne n’a demandé au peuple sur quoi il veut voter. On lui dit : ``Votez pour ou contre Maastricht". Mais qui a fait Maastricht ? Ce n’est pas le peuple qui a élaboré ce traité.\\
Il y a la merveilleuse phrase d’Aristote : ``Qui est citoyen ? Est citoyen quelqu’un qui est capable de gouverner et d’être gouverné." Il y a des millions de citoyens en France. Pourquoi ne seraient-ils pas capables de gouverner ? Parce que toute la vie politique vise précisément à le leur désapprendre, à les convaincre qu’il y a des experts à qui il faut confier les affaires. Il y a donc une contre-éducation politique. Alors que les gens devraient s’habituer à exercer toutes sortes de responsabilités et à prendre des initiatives, ils s’habituent à suivre ou à voter pour des options que d’autres leur présentent. Et comme les gens sont loin d’être idiots, le résultat, c’est qu’ils y croient de moins en moins et qu’ils deviennent cyniques.\\
Dans les sociétés modernes, depuis les révolutions américaine (1776) et française (1789) jusqu’à la seconde guerre mondiale (1945) environ, il y avait un conflit social et politique vivant. Les gens s’opposaient, manifestaient pour des causes politiques. Les ouvriers faisaient grève, et pas toujours pour de petits intérêts corporatistes. Il y avait de grandes questions qui concernaient tous les salariés. Ces luttes ont marqué ces deux derniers siècles.\\
On observe un recul de l’activité des gens. C’est un cercle vicieux. Plus les gens se retirent de l’activité, plus quelques bureaucrates, politiciens, soi-disant responsables, prennent le pas. Ils ont une bonne justification : ``Je prends l’initiative parce que les gens ne font rien." Et plus ils dominent, plus les gens se disent : ``C’est pas la peine de s’en mêler, il y en a assez qui s’en occupent, et puis, de toute façon, on n’y peut rien."\\
La seconde raison, liée à la première, c’est la dissolution des grandes idéologies politiques, soit révolutionnaires, soit réformistes, qui voulaient vraiment changer des choses dans la société. Pour mille et une raisons, ces idéologies ont été déconsidérées, ont cessé de correspondre aux aspirations, à la situation de la société, à l’expérience historique. Il y a eu cet énorme événement qu’est l’effondrement de l’URSS en 1991 et du communisme. Une seule personne, parmi les politiciens --- pour ne pas dire les politicards --- de gauche, a-t-elle vraiment réfléchi sur ce qui s’est passé ? Pourquoi cela s’est-il passé et qui en a, comme on dit bêtement, tiré des leçons ? Alors qu’une évolution de ce type, d’abord dans sa première phase --- l’accession à la monstruosité, le totalitarisme, le Goulag, etc. --- et ensuite dans l’effondrement, méritait une réflexion très approfondie et une conclusion sur ce qu’un mouvement qui veut changer la société peut faire, doit faire, ne doit pas faire, ne peut pas faire. Rien !\\
Et que font beaucoup d’intellectuels ? Ils ont ressorti le libéralisme pur et dur du début du XIXe siècle, qu’on avait combattu pendant cent cinquante ans, et qui aurait conduit la société à la catastrophe. Parce que, finalement, le vieux Marx n’avait pas entièrement tort. Si le capitalisme avait été laissé à lui-même, il se serait effondré cent fois. Il y aurait eu une crise de surproduction tous les ans. Pourquoi ne s’est-il pas effondré ? Parce que les travailleurs ont lutté, ont imposé des augmentations de salaire, ont créé d’énormes marchés de consommation interne. Ils ont imposé des réductions du temps de travail, ce qui a absorbé tout le chômage technologique. On s’étonne maintenant qu’il y ait du chômage. Mais depuis 1940 le temps de travail n’a pas diminué.\\
Les libéraux nous disent : ``Il faut faire confiance au marché." Mais les économistes académiques eux-mêmes ont réfuté cela dès les années 30. Ces économistes n’étaient pas des révolutionnaires, ni des marxistes ! Ils ont montré que tout ce que racontent les libéraux sur les vertus du marché, qui garantirait la meilleure allocation possible des ressources, la distribution des revenus la plus équitable, ce sont des aberrations ! Tout cela a été démontré. Mais il y a cette grande offensive économico-politique des couches gouvernantes et dominantes qu’on peut symboliser par les noms de M. Reagan et de Mme Thatcher, et même de François Mitterrand ! Il a dit : ``Bon, vous avez assez rigolé. Maintenant, on va vous licencier", on va éliminer la ``mauvaise graisse", comme avait dit M. Juppé ! ``Et puis vous verrez que le marché, à la longue, vous garantit le bien-être." A la longue. En attendant, il y a 12,5\% de chômage officiel en France !

\section*{La crise n'est pas une fatalité}

On a parlé d’une sorte de terrorisme de la pensée unique, c’est-à-dire une non-pensée. Elle est unique en ce sens qu’elle est la première pensée qui soit une non-pensée intégrale. Pensée unique libérale à laquelle nul n’ose s’opposer. Qu’était l’idéologie libérale à sa grande époque ? Vers 1850, c’était une grande idéologie parce qu’on croyait au progrès. Ces libéraux-là pensaient qu’avec le progrès il y aurait élévation du bien-être économique. Même quand on ne s’enrichissait pas, dans les classes exploitées, on allait vers moins de travail, vers des travaux moins pénibles : c’était le grand thème de l’époque. Benjamin Constant le dit : ``Les ouvriers ne peuvent pas voter parce qu’ils sont abrutis par l’industrie [il le dit carrément, les gens étaient honnêtes à l’époque !], donc il faut un suffrage censitaire."\\
Par la suite, le temps de travail a diminué, il y a eu l’alphabétisation, l’éducation, des espèces de Lumières qui ne sont plus les Lumières subversives du XVIIIe siècle mais des Lumières qui se diffusent tout de même dans la société. La science se développe, l’humanité s’humanise, les sociétés se civilisent et petit à petit on arrivera à une société où il n’y aura pratiquement plus d’exploitation, où cette démocratie représentative tendra à devenir une vraie démocratie.\\
Mais cela n’a pas marché ! Donc les gens ne croient plus à cette idée. Aujourd’hui ce qui domine, c’est la résignation ; même chez les représentants du libéralisme. Quel est le grand argument, en ce moment ? ``C’est peut-être mauvais mais l’autre terme de l’alternative était pire." Et c’est vrai que cela a glacé pas mal les gens. Ils se disent : ``Si on bouge trop, on va vers un nouveau Goulag." Voilà ce qu’il y a derrière cet épuisement idéologique et on n’en sortira que si vraiment il y a une résurgence d’une critique puissante du système. Et une renaissance de l’activité des gens, d’une participation des gens.\\
Çà et là, on commence quand même à comprendre que la ``crise" n’est pas une fatalité de la modernité à laquelle il faudrait se soumettre, ``s’adapter" sous peine d’archaïsme. On sent frémir un regain d’activité civique. Alors se pose le problème du rôle des citoyens et de la compétence de chacun pour exercer les droits et les devoirs démocratiques dans le but --- douce et belle utopie --- de sortir du conformisme généralisé.\\
Pour en sortir, faut-il s’inspirer de la démocratie athénienne ? Qui élisait-on à Athènes ? On n’élisait pas les magistrats. Ils étaient désignés par tirage au sort ou par rotation. Pour Aristote, souvenez-vous, un citoyen, c’est celui qui est capable de gouverner et d’être gouverné. Tout le monde est capable de gouverner, donc on tire au sort. La politique n’est pas une affaire de spécialiste. Il n’y a pas de science de la politique. Il y a une opinion, la doxa des Grecs, il n’y a pas d’épistémè\footnote{Savoir théoriquement fondé, science.}.\\
L’idée selon laquelle il n’y a pas de spécialiste de la politique et que les opinions se valent est la seule justification raisonnable du principe majoritaire. Donc, chez les Grecs, le peuple décide et les magistrats sont tirés au sort ou désignés par rotation. Pour les activités spécialisées --- construction des chantiers navals, des temples, conduite de la guerre ---, il faut des spécialistes. Ceux-là, on les élit. C’est cela, l’élection. Election veut dire ``choix des meilleurs". Là intervient l’éducation du peuple. On fait une première élection, on se trompe, on constate que, par exemple, Périclès est un déplorable stratège, eh bien on ne le réélit pas ou on le révoque.\\
Mais il faut que la doxa soit cultivée. Et comment une doxa concernant le gouvernement peut-elle être cultivée ? En gouvernant. Donc la démocratie --- c’est important --- est une affaire d’éducation des citoyens, ce qui n’existe pas du tout aujourd’hui.

\section*{``Se reposer ou être libre"}

Récemment, un magazine a publié une statistique indiquant que 60\% des députés, en France, avouent ne rien comprendre à l’économie. Des députés qui décident tout le temps ! En vérité, ces députés, comme les ministres, sont asservis à leurs techniciens. Ils ont leurs experts, mais ils ont aussi des préjugés ou des préférences. Si vous suivez de près le fonctionnement d’un gouvernement, d’une grande bureaucratie, vous voyez que ceux qui dirigent se fient aux experts, mais choisissent parmi eux ceux qui partagent leurs opinions. C’est un jeu complètement stupide et c’est ainsi que nous sommes gouvernés.\\
Les institutions actuelles repoussent, éloignent, dissuadent les gens de participer aux affaires. Alors que la meilleure éducation en politique, c’est la participation active, ce qui implique une transformation des institutions qui permette et incite à cette participation.\\
L’éducation devrait être beaucoup plus axée vers la chose commune. Il faudrait comprendre les mécanismes de l’économie, de la société, de la politique, etc. Les enfants s’ennuient en apprenant l’histoire alors que c’est passionnant. Il faudrait enseigner une véritable anatomie de la société contemporaine, comment elle est, comment elle fonctionne. Apprendre à se défendre des croyances, des idéologies.\\
Aristote a dit : ``L’homme est un animal qui désire le savoir." C’est faux. L’homme est un animal qui désire la croyance, qui désire la certitude d’une croyance, d’où l’emprise des religions, des idéologies politiques. Dans le mouvement ouvrier, au départ, il y avait une attitude très critique. Prenez le deuxième couplet de L’Internationale, le chant de la Commune : ``Il n’est pas de Sauveur suprême, ni Dieu - exit la religion - ni César, ni tribun" - exit Lénine !\\
Aujourd’hui, même si une frange cherche toujours la foi, les gens sont devenus beaucoup plus critiques. C’est très important. La scientologie, les sectes, ou le fondamentalisme, c’est dans d’autres pays, pas chez nous, pas tellement. Les gens sont devenus beaucoup plus sceptiques. Ce qui les inhibe aussi pour agir.\\
Périclès dans le discours aux Athéniens dit : ``Nous sommes les seuls chez qui la réflexion n’inhibe pas l’action." C’est admirable ! Il ajoute : ``Les autres, ou bien ils ne réfléchissent pas et ils sont téméraires, ils commettent des absurdités, ou bien, en réfléchissant, ils arrivent à ne rien faire parce qu’ils se disent, il y a le discours et il y a le discours contraire." Actuellement, on traverse une phase d’inhibition, c’est sûr. Chat échaudé craint l’eau froide. Il ne faut pas de grands discours, il faut des discours vrais.\\
De toute façon il y a un irréductible désir. Si vous prenez les sociétés archaïques ou les sociétés traditionnelles, il n’y a pas un irréductible désir, un désir tel qu’il est transformé par la socialisation. Ces sociétés sont des sociétés de répétition. On dit par exemple : ``Tu prendras une femme dans tel clan ou dans telle famille. Tu auras une femme dans ta vie. Si tu en as deux, ou deux hommes, ce sera en cachette, ce sera une transgression. Tu auras un statut social, ce sera ça et pas autre chose."\\
Or, aujourd’hui, il y a une libération dans tous les sens du terme par rapport aux contraintes de la socialisation des individus. On est entré dans une époque d’illimitation dans tous les domaines, et c’est en cela que nous avons le désir d’infini. Cette libération est en un sens une grande conquête. Il n’est pas question de revenir aux sociétés de répétition. Mais il faut aussi --- et c’est un très grand thème --- apprendre à s’autolimiter, individuellement et collectivement. La société capitaliste est une société qui court à l’abîme, à tous points de vue, car elle ne sait pas s’autolimiter. Et une société vraiment libre, une société autonome, doit savoir s’autolimiter, savoir qu’il y a des choses qu’on ne peut pas faire ou qu’il ne faut même pas essayer de faire ou qu’il ne faut pas désirer.\\
Nous vivons sur cette planète que nous sommes en train de détruire, et quand je prononce cette phrase je songe aux merveilles, je pense à la mer Egée, je pense aux montagnes enneigées, je pense à la vue du Pacifique depuis un coin d’Australie, je pense à Bali, aux Indes, à la campagne française qu’on est en train de désertifier. Autant de merveilles en voie de démolition. Je pense que nous devrions être les jardiniers de cette planète. Il faudrait la cultiver. La cultiver comme elle est et pour elle-même. Et trouver notre vie, notre place relativement à cela. Voilà une énorme tâche. Et cela pourrait absorber une grande partie des loisirs des gens, libérés d’un travail stupide, productif, répétitif, etc. Or cela est très loin non seulement du système actuel mais de l’imagination dominante actuelle. L’imaginaire de notre époque, c’est celui de l’expansion illimitée, c’est l’accumulation de la camelote --- une télé dans chaque chambre, un micro-ordinateur dans chaque chambre ---, c’est cela qu’il faut détruire. Le système s’appuie sur cet imaginaire-là.\\
La liberté, c’est très difficile. Parce qu’il est très facile de se laisser aller. L’homme est un animal paresseux. Il y a une phrase merveilleuse de Thucydide : ``Il faut choisir : se reposer ou être libre." Et Périclès dit aux Athéniens : ``Si vous voulez être libres, il faut travailler." Vous ne pouvez pas vous reposer. Vous ne pouvez pas vous asseoir devant la télé. Vous n’êtes pas libres quand vous êtes devant la télé. Vous croyez être libres en zappant comme un imbécile, vous n’êtes pas libres, c’est une fausse liberté. La liberté, c’est l’activité. Et la liberté, c’est une activité qui en même temps s’autolimite, c’est-à-dire sait qu’elle peut tout faire mais qu’elle ne doit pas tout faire. C’est cela le grand problème de la démocratie et de l’individualisme.


