\chapter{Pourquoi je ne suis pas un anarchiste}
\chapterauthor{Gregory Hill}

\emph{Cet article a été publié pour la première fois dans No Governor, Vol. I, No. 2, 1975.\\}

Il y a environ cinq ans, je me considérais comme un anarchiste (anarcho-pacifiste, en particulier), parce que je crois que la plus haute autorité disponible pour tout individu est sa propre expérience sincère et que toute autre autorité ne fournit au mieux que des informations indirectes.\\
Je n'ai pas changé d'avis à ce sujet, mais j'ai cessé de me qualifier comme anarchiste. La raison en est simple et basique : \textsc{trop de putain de règles}.\\
OK, c'est une blague. Mais c'est une blague \textsc{vraie}. L'incompatibilité n'est pas entre ma position et certaines théories anarchistes, mais entre ma position et la position de la plupart de ceux qui utilisent l'étiquette ``anarchiste.''
Il semble que la Règle Numéro Un de l'anarchie, telle qu'elle est comprise par les autoritaires et par la plupart de ceux qui se disent anarchistes, est qu'un gouvernement est un ennemi. La Règle Numéro Deux est que pour obtenir la liberté, l'individu est politiquement ou moralement ou d'une manière ou d'une autre obligé de combattre cet ennemi.\\
À mon avis, ces règles représentent une position qu'il vaudrait mieux qualifier d'anti-archie. Le préfixe ``a'' signifie ``sans'' et n'implique pas nécessairement ``contre''. Il y a un parallèle exact avec le mot athéisme --- il est habituellement utilisé et compris, par ceux qui sont pour et contre, comme si le mot était anti-théisme.\\
Je peux respecter la position anti-archiste, mais je ne la partage pas. Le gouvernement n'est pas mon ennemi parce qu'il n'y a pas de gouvernement. OK, c'est une autre blague, mais c'est encore une blague \textsc{vraie}. Je sais bien qu'il y a des gens armés qui restreignent ma liberté de décision, et je connais des groupes de gens qui perçoivent des impôts auprès de moi, et tout le reste de ce gouvernement le perçoit de la même manière que je perçois (par exemple) un gros rocher sur mon chemin qui m'oblige à faire un pas de côté et à m'adapter. Franchement, je ne crois pas non plus aux rochers --- Je me contente de faire un pas et de m'adapter (ce qui est en fait plus facile que de croire en eux). Je pense qu'il y a une grande différence de niveau entre (a) répondre existentiellement à un phénomène et (b) le conceptualiser comme un ``ennemi''. Si tout ce qui, dans l'univers, a déjoué mon dessein est mon ennemi, rien ne peut être mon ami --- et cela m'exclut moi-même. Mais, tout de même, je respecte la position anti-archiste. Après tout, si l'on perçoit un phénomène comme étant un ennemi, alors on serait un fou de faire autre chose que de se défendre.\\
Le gros de cet essai est de pinailler sur les étiquettes. Pourtant, je me sens libre de pinailler, et en tout cas ce que j'essaie de faire, c'est d'éviter que d'autres pensent que je suis en guerre avec certaines personnes simplement parce que ces personnes pensent qu'elles sont un gouvernement et font tout ce qu'elles peuvent pour m'imposer leurs notions par la force.\\
Je ne suis pas en guerre avec eux ni avec les rochers. Et dans la mesure où quelqu'un pense qu'un anarchiste est celui qui est censé faire une chose ou une autre, il y a trop de règles pour moi et au diable tout ce bordel.