\chapter{Stop the Rise of Insignificance}
\chapterauthor{Cornelius Castoriadis}

\emph{This text comes from an interview by Daniel Mermet for the radio program ``L\`a-bas si j'y suis" from 1996. Translated from French by Griffensteed the 5$^\text{lth}$.\\}

What characterizes the contemporary world are, of course, crises, contradictions, oppositions, fractures, but what strikes me the most is insignificance. Let us consider the feud between the right wing and the left wing. It has lost its meaning. Both of them say the same thing. Since 1983, the French socialists conducted a policy, then Mr. Balladur conducted the same policy; the socialists came back, they conducted, with Pierre Bérégovoy, the same policy; Mr. Balladur came back, he conducted the same policy; Mr. Chirac won the 1995 election saying "I'm going to do something else" and he conducted the same policy.\\
Political leaders are powerless. The only thing they can do is to go with the flow, which is to apply the trendy ultraliberal policy. The Socialists did no other thing once they returned to power. They are not political people, but politicians in the sense of micro-politicians. People who hunt votes by any means. They have no program. Their goal is to stay in power or return to power, and for that they are capable of anything.\\
There is an intrinsic link between this kind of nullity of politics, this becoming null of politics and this insignificance in other fields, in the arts, in philosophy or in literature. That is the spirit of the times. Everything conspires to extend insignificance.\\
Politics is a strange profession. Because it presupposes two skills that have no intrinsic relationship. The first is to gain power. If you don't come to power, you can have the best ideas in the world, it's useless; which implies an art of coming to power. The second ability is, once you're in power, to know how to govern.\\
There is no guarantee that someone who knows how to govern will be able to gain power. In the absolute monarchy, to gain power it was necessary to flatter the king, to be in the good graces of Madame de Pompadour. Today in our ``pseudo-democracy", coming to power means being telegenic, sensing public opinion.\\
I say ``pseudo-democracy" because I have always thought that the so-called representative democracy is not a true democracy. Jean-Jacques Rousseau had already said it: the English believe they are free because they elect representatives every five years, but they are free one day per five years, election day, that's all. Not that the election is rigged, not that they cheat in the polls. It's rigged because the options are set in advance. No one has asked the people what they want to vote on. We say to them: ``Vote for or against Maastricht". But who did Maastricht? It is not the people who made this treaty.\\
There is this wonderful phrase from Aristotle: ``Who is a citizen? A citizen is someone who is able to govern and be governed." There are millions of citizens in France. Why shouldn't they be able to govern? Because all political life aims precisely to make them unlearn it, to convince them that there are experts to whom to entrust business. So there is political counter-education. While people should get used to exercising all kinds of responsibilities and taking initiatives, they get used to following or voting for options that others present to them. And since people are far from  being stupid, the result is that they believe less and less and become cynical.\\
In modern societies, from the American revolution (1776) and the French revolution (1789) to the Second World War (1945), there was a thriving social and political conflict. People were opposed, demonstrated for political causes. The workers went on strike, and not always for small corporate interests. There were big questions that concerned all employees. These struggles have marked the last two centuries.\\
We are witnessing a decline in people's activity. It's a vicious circle. The more people withdraw from the activity, the more a few bureaucrats, politicians, supposedly responsible, take the lead. They have a good justification: ``I take the initiative because people do nothing." And the more they dominate, the more people say, ``It's no use getting involved, there's enough of them, and even then, we can't do anything about it anyway."\\
The second reason, related to the first, is the dissolution of the great political ideologies, either revolutionary or reformist, which really wanted to change things in society. For a thousand and one reasons, these ideologies have been discredited, have ceased to correspond to aspirations, to the situation of society, to historical experience. There was the massive event of the collapse of the USSR and communism in 1991. Has one person among the politicians --- not to say the schemers --- on the left, really thought about what happened? Why did this happen and who, as we say, has learned from it? While an evolution of this type, in its first phase --- the attainment of monstrosity, totalitarianism, the Gulag, etc. --- and then in the collapse, deserved a very thorough reflection and a conclusion on what a movement that wants to change society can do, must do, must not do, cannot do. Nothing!\\
And what do many intellectuals do? They brought out the strict liberalism of the early 19th century, that we fought for 150 years, and which would have led society to catastrophe. Because, in the end, the old Marx wasn't entirely wrong. If capitalism had been left to its own devices, it would have collapsed a hundred times. There would have been an overproduction crisis every year. Why didn't it collapse? Because workers fought, imposed wage increases, created huge internal consumer markets. They imposed reductions in working time, which absorbed all the technological unemployment. Now we are surprised that there is unemployment. But since 1940 working time hasn't decreased.\\
The Liberals tell us, "You have to trust the market." But academic economists themselves refuted this as early as the 1930s. These economists were not revolutionaries, nor were they Marxists! They have shown that all that the Liberals are saying about the virtues of the market, which would guarantee the best possible allocation of resources, the most equitable distribution of income, is absolute nonsense! All this has been demonstrated. But there is this great economic-political offensive of the governing and dominant layers that can be symbolized by the names of Mr. Reagan and Mrs. Thatcher, and even François Mitterrand! He said, "Okay, you've had enough fun. Now we'll fire you," we'll eradicate the "bad fat," as Mr. Juppé said. "And then you'll see that the market, in the long run, guarantees your well-being." In the long run. Meanwhile, there's 12.5\% official unemployment in France!

\section*{The Crisis is not a Fatality}

There has been talk of a kind of one-track thinking terrorism, that is, a non-thinking terrorism. It is one-track in the sense that it is the first thought that is an integral non-thought. A unique liberal thought that no one dares to oppose. What was Liberal ideology in its heyday? Around 1850, it was a great ideology because we believed in progress. Those Liberals thought that with progress there would be an increase in economic well-being. Even when we did not get richer, in the exploited classes, we went towards less workload, towards less arduous work: this was the great theme of the time. Benjamin Constant said: "Workers can't vote because they're dumbfounded by the industry [he says it bluntly, people were honest at the time!], so you need a census suffrage."\\
Afterwards, the working time decreased, there was literacy, education, a kind of Enlightenment that is no longer the subversive Enlightenment of the 18th century, but Enlightenment that spreads throughout society. Science develops, humanity becomes human, societies become more civilized, and little by little we will arrive at a society where there will be practically no more exploitation, where this representative democracy will tend to become a true democracy.\\
But it didn't work! So people no longer believe that idea. Today what dominates is resignation, even among the representatives of liberalism. What's the big argument right now? ``It may be bad, but the other alternative term was worse." And it's true that it frightened people a lot. They say to themselves, ``If we move too much, we go to a new Gulag." That is what is behind this ideological exhaustion and we will only come out of it if there really is a resurgence of a powerful criticism of the system. And a renaissance of people's activity, people's participation.\\
Here and there, we begin to understand that the ``crisis" is not a fatality of modernity to which we would have to subject ourselves, ``to adapt'' under penalty of archaism. We feel a shiver of renewed civic activity. Then the problem arises of the role of citizens and the competence of each to exercise democratic rights and duties with the aim --- sweet and beautiful utopia --- of getting out of generalized conformism.\\
To get out of it, should we be inspired by Athenian democracy? Who was elected in Athens? They didn't elect magistrates. They were designated by a random draw or by rotation. For Aristotle, remember, a citizen is one who is capable of governing and being governed. Everyone is capable of governing, so we do random draws. Politics is not a matter for specialists. There is no political science. There's an opinion, the Greek doxa, there's no episteme\footnote{Theoretically founded knowledge, science.}.\\
The idea that there is no polics expert and that all opinions are equal is the only reasonable justification for the majority principle. Thus, among the Greeks, the people decide and the magistrates are designated randomly or by rotation. For specialized activities --- shipyard construction, temple construction, warfare --- specialists are needed. Them, they do get elected. That is election. Election means ``choice of the best". Here is where education of the people comes in. We do a first election, we're wrong, we find that, for example, Pericles is a deplorable strategist, well, we don't re-elect him or we revoke him.\\
But the doxa must be cultivated. And how can a doxa be cultivated for the government? By governing. Therefore democracy --- it is important --- is a matter of educating citizens, which does not exist at all today.

\section*{``Rest or Be Free"}

Recently, a journal published a statistic indicating that 60\% of MPs, in France, admit they understand nothing about the economy. MPs who decide all the time! In truth, these members, like ministers, are subjugated to their technicians. They have their experts, but they also have prejudices or preferences. If you follow closely the functioning of a government, of a large bureaucracy, you see that those who lead rely on experts, but choose among them those who share their opinions. It's a completely stupid game and that's how we're governed.\\
Today's institutions repel, distance and dissuade people from participating in business. While the best education in politics is active participation, which implies institutional transformation that enables and encourages that participation.\\
Education should be much more focused on the common thing. We should understand the mechanisms of the economy, society, politics, etc. Children get bored learning history when it's fascinating. We should teach a true anatomy of contemporary society, how it is, how it works. Learn to defend beliefs, ideologies.\\
Aristotle said, "Man is an animal that desires to know." That is not true. Man is an animal that desires belief, that desires the certainty of belief, hence the hold of religions, of political ideologies. In the labour movement, at first, there was a very critical attitude. Take the second verse of The International, the song of Paris Commune: ``There is no Supreme Savior, no God --- exit religion --- no Caesar, no tribune" --- exit Lenin!\\
Today, even if a minority still seeks faith, people have become much more critical. This is very important. Scientology, sects, or fundamentalism, it's in other countries, not here, not so much. People have become much more skeptical. Which also hinders their will to act.\\
Pericles in the speech to the Athenians said: ``We are the only ones in whom reflection does not inhibit action." That's admirable! He adds: ``The others, either they do not think and they are reckless, they commit absurdities, or, by thinking, they don't manage to do anything because they say to themselves, there is discourse and there is the opposite discourse''. Right now, we're definitely going through an inhibition phase. Once bitten, twice shy. We don't need big speeches, we need true speeches.\\
Anyway, there is an irreducible desire. If you take archaic societies or traditional societies, there is not an irreducible desire, a desire such as it is transformed by socialization. These societies are societies of repetition. For example, we say: ``You will take a woman from such a clan or from such a family. You'll have a wife in your life. If you have two, or two men, it'll be secret, it'll be a transgression. You'll have a social status, it'll be that and nothing else."\\
Today, however, there is a liberation, in every sense of the word, from the constraints of the socialization of individuals. We have entered an age of unlimitedness in all fields, and it is due to this that we have the desire for infinity. This liberation is in a sense a great conquest. There is no question of returning to societies of repetition. But we must also --- and this is a very great theme-- learn to limit ourselves, individually and collectively. The capitalist society is a society that runs into the abyss, from every point of view, because it does not know how to limit itself. And a truly free society, an autonomous society, has to know how to limit itself, how to know that there are things you can't do or even try to do or want.\\
We live on this planet that we are destroying, and when I say this sentence I think of the wonders, I think of the Aegean Sea, I think of the snow-covered mountains, I think of the view of the Pacific from a corner of Australia, I think of Bali, India, the French countryside that is being desertified. So many wonders in the process of demolition. I think we should be the gardeners of this planet. It should be cultivated. Grow it as it is and for itself. And find our life, our place in it. That is a huge task. And it could absorb a large part of people's leisure time, freed from a stupid, productive, repetitive work, etc. This is not only far away from the current system but also from the current dominant imagination. The imagination of our time is that of unlimited expansion, the accumulation of junk --- a TV in every room, a microcomputer in every room ---, that is what we must destroy. The system relies on that imagination.\\
Freedom is very difficult. Because it's very easy to let yourself go. Man is a lazy animal. There's a wonderful phrase from Thucydides: ``You have to choose: rest or be free." And Pericles said to the Athenians, ``If you want to be free, you must work." You can't rest now. You can't sit in front of the TV. You're not free when you're watching TV. You think you are free by flicking through channels like an idiot, you are not free, it is a false freedom. Freedom is activity. And freedom is an activity that at the same time limits itself, that is, it knows that it can do everything but that it must not do everything. That is the great problem of democracy and individualism.


